\documentclass{article}
\usepackage{amsmath}
\usepackage{enumitem}

\title{Counting and Probability Week 5 Writing Problem}
\author{mxsail} % Lower case to align with mxsail's AoPS username
\date{\today}

\begin{document}
\maketitle

\section*{Problem Statement}
In a single-file queue of $n$ people with distinct heights, define a blocker to be someone who is either taller than the person standing immediately behind them, or the last person in the queue. For example, suppose that Ashanti has height $a,$ Blaine has height $b,$ Charlie has height $c,$ Dakota has height $d,$ and Elia has height $e,$ and that $a<b<c<d<e.$ If these five people lined up in the order Ashanti, Elia, Charlie, Blaine, Dakota (from front to back), then there would be three blockers: Elia, Charlie, and Dakota. For integers $n \ge 1$ and $k \ge 0,$ let $Q(n,k)$ be the number of ways that $n$ people can queue up such that there are exactly $k$ blockers.

\begin{enumerate}[label=(\alph*)]
    \item Show that 
    \[Q(3,2)= 2 \cdot Q(2,2)+ 2 \cdot Q(2,1).\]
    \item Show that for $n \ge 2$ and $k \ge 1,$
    \[Q(n,k)=k \cdot Q(n-1,k)+(n-k+1) \cdot Q(n-1,k-1).\]
    (You can assume that $Q(1,1)=1,$ and that $Q(n,0)=0$ for all $n.$)
\end{enumerate}

\section*{Solution}
\subsection*{(a)}
To compute the values of \( Q(3,2) \), \( Q(2,2) \), and \( Q(2,1) \),
we first list all possible arrangements of 3 people with distinct heights,
say A, B, and C, where A < B < C. We can assume, without loss of generality,
that the heights are 1, 2, and 3.
The possible arrangements of A, B, and C are:
\begin{itemize}
    \item ABC \( \implies 1,2,3 \) (1 blocker: C)
    \item ACB \( \implies 1,3,2 \) (2 blockers: C, B)
    \item BAC \( \implies 2,1,3 \) (2 blockers: B, C)
    \item BCA \( \implies 2,3,1 \) (2 blockers: C, A)
    \item CAB \( \implies 3,1,2 \) (2 blockers: C, B)
    \item CBA \( \implies 3,2,1 \) (3 blockers: C, B, A)
\end{itemize}
We see that there is one arrangement with 1 blocker (ABC),
four arrangements with 2 blockers (ACB, BAC, BCA, CAB),
and one arrangement with 3 blockers (CBA).
Therfore:
\[Q(3,2) = 4 \]
\[Q(2,2) = 1 \]
\[Q(2,1) = 1 \]
Subsistuting for \( Q(3,2) \), \( Q(2,2) \), and \( Q(2,1) \):
\[ 4 = 2 \cdot 1 + 2 \cdot 1 \]
Thus, the equation holds.

\subsection*{(b)}
To prove the recurrence relation
\[Q(n,k)=k \cdot Q(n-1,k)+(n-k+1) \cdot Q(n-1,k-1),\]
We work backwards from Q(n-1,k) and Q(n-1,k-1) to Q(n,k).
Consider a queue of \( n-1 \) people with \( k \) blockers.
When we add the \( n^{th} \) person to the queue, who is the tallest person,
there are two possible cases:
\begin{enumerate}
    \item The new person does not affect the number of blockers. In this case,
    the new person must be placed behind one of the existing blockers
    in the queue. There are \( k \) blockers in the queue of \( n-1 \)
    people, so there are \( k \) possible positions to place the new
    person. This contributes \( k \cdot Q(n-1,k) \) arrangements.
    \item The new person is a blocker. In this case, the new person
    must be placed in a position that is not immediately behind an existing blocker.
    The set of possible positions
    to place the new person is the total number of positions \( n-1 \)
    minus the number of positions immediately behind existing blockers \( k-1 \).
    this gives us \( n-1-(k-1)= n\) possible positions. However, the new person
    can also be placed at the start of the queue, always as a blocker,
    becuase they are the tallest person.
    Therefore, there are \( n-k+1 \) possible positions to place the new person.
    This contributes \( (n-k+1) \cdot Q(n-1,k-1) \) arrangements.
\end{enumerate}
Combining both cases, we have:
\[Q(n,k)=k \cdot Q(n-1,k)+(n-k+1) \cdot Q(n-1,k-1).\]
Describing the ways to form a queue of \( n \) people with \( k \) blockers
by adding the \( n^{th} \) person to a queue of \( n-1 \) people with either
\( k \) or \( k-1 \) blockers.

\end{document}