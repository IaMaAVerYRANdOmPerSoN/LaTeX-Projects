\documentclass{article}
\usepackage{amsmath}

\title{Week 2 Writing Problem}
\author{Mxsail,\\
The Art of Problem Solving,\\
Intermediate Counting and Probability}
\date{\today}

\begin{document}
\maketitle

\section*{Problem Statement}
A standard $8 \times 8$ chessboard has $64$ unit squares.

Eight rooks are randomly placed on different squares of a chessboard. A rook is said to attack all of the squares in its row and its column.

Compute the probability that every square is occupied or attacked by at least one rook. You may express your answer in terms of binomial coefficients.

\section*{Solution}

We can consider the problem in terms of rows (ranks) and columns (files). There are two main cases to consider:

\begin{enumerate}
    \item All rooks are on different ranks.
    \item All rooks are on different files.
\end{enumerate}

\subsection*{Case 1: Rooks on Different Ranks}
For the first rank, there are $8$ possible positions for the rook.  
For the second rank, there are $8$ possible positions, and so on for all $8$ ranks.  

Thus, the total number of valid placements with all rooks on different ranks is
\[
8^8.
\]

\subsection*{Case 2: Rooks on Different Files}
By symmetry, the same argument applies to columns (files). There are also
\[
8^8
\]
valid placements with all rooks on different files.

\subsection*{Accounting for Overlap}
Some arrangements are counted in both cases — specifically, the arrangements where all rooks are on distinct ranks \emph{and} distinct files. These correspond exactly to permutations of the $8$ rows and $8$ columns, giving $8!$ overcounted arrangements.

\subsection*{Probability}
The total number of valid placements is therefore
\[
2 \cdot 8^8 - 8!.
\]

The total number of ways to place $8$ rooks on the board is
\[
\binom{64}{8}.
\]

Hence, the probability that every square is occupied or attacked is
\[
\frac{2 \cdot 8^8 - 8!}{\binom{64}{8}} \approx 0.00759.
\]

\end{document}