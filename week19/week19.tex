\documentclass{article}
\usepackage{amsmath}
\usepackage{amssymb}
\usepackage{enumitem}

\title{Week 19 Writing Problem}
\author{mxsail}
\date{\today}

\begin{document}

\maketitle

\section*{Problem Statement}

Let
\[
f(x) = \left\lfloor x \left\lfloor x \right\rfloor \right\rfloor \quad \text{for } x \ge 0.
\]

\begin{enumerate}[label=(\alph*)]
    \item Find all $x \ge 0$ such that $f(x) = 1$.
    \item Find all $x \ge 0$ such that $f(x) = 3$.
    \item Find all $x \ge 0$ such that $f(x) = 5$.
    \item Find the number of possible values of $f(x)$ for $0 \le x \le 10$.
\end{enumerate}

\section*{Solution}

Let $n = \lfloor x \rfloor$, so $n \le x < n+1$ and $n$ is a nonnegative integer. Then
\[
f(x) = \left\lfloor x \left\lfloor x \right\rfloor \right\rfloor = \lfloor x n \rfloor.
\]

\subsection*{(a) $f(x) = 1$}

We want $\lfloor x n \rfloor = 1$.

\begin{align*}
&\text{If } n = 0: && x n = 0 \implies f(x) = 0. \\
&\text{If } n = 1: && x \in [1,2),\ x n = x \in [1,2) \implies \lfloor x n \rfloor = 1. \\
&\text{If } n \ge 2: && x n \ge 2n > 1.
\end{align*}

Thus, the solution is
\[
\boxed{x \in [1,2)}
\]

\subsection*{(b) $f(x) = 3$}

We want $\lfloor x n \rfloor = 3$.

\begin{align*}
&\text{If } n = 0: && x n = 0. \\
&\text{If } n = 1: && x \in [1,2),\ x n \in [1,2) \implies \lfloor x n \rfloor = 1. \\
&\text{If } n = 2: && x \in [2,3),\ x n \in [4,6) \implies \lfloor x n \rfloor \ge 4. \\
&\text{If } n = 3: && x \in [3,4),\ x n \in [9,12) \implies \lfloor x n \rfloor \ge 9.
\end{align*}

There is no $x \ge 0$ such that $f(x) = 3$.

\[
\boxed{\text{No solution}}
\]

\subsection*{(c) $f(x) = 5$}

Try $n = 2$:
\[
x \in [2,3),\ x n \in [4,6)
\]
We want $x n \in [5,6)$, so $x \in [2.5,3)$.

Thus,
\[
\boxed{x \in [2.5,3)}
\]

\subsection*{(d) Number of possible values of $f(x)$ for $0 \le x \le 10$}

For each integer $n$ from $0$ to $10$:
\begin{align*}
n = 0: &\quad x \in [0,1),\ f(x) = 0 \\
n = 1: &\quad x \in [1,2),\ f(x) = 1 \\
n = 2: &\quad x \in [2,3),\ f(x) = 4,5 \\
n = 3: &\quad x \in [3,4),\ f(x) = 9,10,11 \\
n = 4: &\quad x \in [4,5),\ f(x) = 16,17,18,19 \\
n = 5: &\quad x \in [5,6),\ f(x) = 25,26,27,28,29 \\
n = 6: &\quad x \in [6,7),\ f(x) = 36,37,38,39,40,41 \\
n = 7: &\quad x \in [7,8),\ f(x) = 49,50,51,52,53,54,55 \\
n = 8: &\quad x \in [8,9),\ f(x) = 64,65,66,67,68,69,70,71 \\
n = 9: &\quad x \in [9,10],\ f(x) = 81,82,83,84,85,86,87,88,89 \\
n = 10: &\quad x = 10,\ f(10) = 100
\end{align*}

Listing all values:
\[
\begin{aligned}
&0,\,1,\,4,\,5,\,9,\,10,\,11,\,16,\,17,\,18,\,19,\,25,\,26,\,27,\,28,\,29,\,36,\,37,\,38,\,39,\,40,\,41,\\
&49,\,50,\,51,\,52,\,53,\,54,\,55,\,64,\,65,\,66,\,67,\,68,\,69,\,70,\,71,\,81,\,82,\,83,\,84,\,85,\,86,\\
&87,\,88,\,89,\,100
\end{aligned}
\]

Counting, we get $47$ values.

\[
\boxed{47}
\]

\end{document}
