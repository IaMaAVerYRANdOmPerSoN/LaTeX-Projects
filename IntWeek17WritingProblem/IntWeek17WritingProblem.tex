\documentclass{article}
\usepackage{amsmath}
\usepackage{amssymb}

\title{Writing Problem for Week 17}
\author{mxsail, \\ Art of Problem Solving}
\date{\today}

\begin{document}

\maketitle

\section*{Problem Statement}
Find all real solutions for $x$ in
\[ 2(2^x- 1) x^2 + (2^{x^2}-2)x = 2^{x+1} -2 . \]

\section*{Solution}

Let us define
\[
f(x) = 2(2^x-1)(x^2-1) + (2^{x^2-1}-1)x.
\]
We claim that the original equation is equivalent to $f(x) = 0$.

\textbf{Step 1: Sign analysis.}

Note that for any real $y$, $2^y-1$ and $y$ have the same sign (since $2^y-1 = 0$ if $y=0$, and $2^y-1 > 0$ if $y > 0$, $2^y-1 < 0$ iff $y < 0$). Thus,
\[
\operatorname{sgn}(2^x-1) = \operatorname{sgn}(x),
\]
\[
\operatorname{sgn}(2^{x^2-1}-1) = \operatorname{sgn}(x^2-1).
\]

Therefore,
\[
\operatorname{sgn}\big(2(2^x-1)(x^2-1)\big) = \operatorname{sgn}(x)\operatorname{sgn}(x^2-1),
\]
\[
\operatorname{sgn}\big((2^{x^2-1}-1)x\big) = \operatorname{sgn}(x^2-1)\operatorname{sgn}(x).
\]

So both terms have the same sign, and thus
\[
\operatorname{sgn}(f(x)) = \operatorname{sgn}(x)\operatorname{sgn}(x^2-1).
\]

But $\operatorname{sgn}(x^2-1) = \operatorname{sgn}(x-1)\operatorname{sgn}(x+1)$, so
\[
\operatorname{sgn}(f(x)) = \operatorname{sgn}(x)\operatorname{sgn}(x-1)\operatorname{sgn}(x+1).
\]

\textbf{Step 2: Zeros of $f(x)$.}

The sign function is zero if and only if its argument is zero. Therefore, $f(x) = 0$ if and only if $x = 0$, $x = 1$, or $x = -1$.

\textbf{Step 3: Conclusion.}

Thus, the only real solutions to the original equation are
\[
\boxed{x = -1,\ 0,\ 1}
\]

\end{document}