\documentclass{article}

\usepackage{float}
\usepackage{amssymb}
\usepackage{amsmath}
\usepackage{amsfonts}
\usepackage{graphicx}
\usepackage{csquotes}

\title{ \vspace{-2.0cm} Basic Graph Transformations} 
\author{Michael Xie \\ Art of Problem Solving}
\date{\today}

\begin{document}

\maketitle

\section{Introduction}
In this article, we will perform and describe some basic graph transformations. We are given a function defined by its graph and asked to plot a transformed version of the given graph. Furthermore, we will explore why these transformations cause the graph to shift as they do.

\subsection{The Problem}

The graph of $y = f(x)$ is shown in the figure below.

\begin{figure}[H]
    \includegraphics[width = \linewidth]{Figure_1.png}
    \caption{Graph of $y = f(x)$}
    \label{fig:Figure_1}
\end{figure}

For each point $(a,b)$ that is on the graph of $y = f(x),$ the point $(2a +1, 3b)$ is plotted, forming the graph of another function $y = g(x).$ As an example, the point $(0,2)$ lies on the graph of $y = f(x),$ so the point $(2 \cdot 0 + 1, 3 \cdot 2) = (1, 6)$ lies on the graph of $y = g(x).$

\begin{itemize}
    \item[(a)] Plot the graph of $y = g(x).$ Include the diagram as part of your solution.
    \item[(b)] Express $g(x)$ in terms of $f(x).$ 
    \item[(c)] Describe the transformations that can be applied to the graph of $y = f(x)$ to obtain the graph of $y = g(x).$ For example, one transformation could be to stretch the graph vertically by a factor of 3. 
\end{itemize}

\section{Solution}

\subsection{Part (a)}
Let's start by plotting the graph, $y = g(x)$. As these are all linear transformations, there is no need to generate or find additional points for a larger sample size. We can use the integer points given:

\begin{figure}[H]
    \includegraphics[width = \linewidth]{Figure_2.png}
    \caption{Graph of $y = g(x)$ vs $y = f(x)$}
    \label{fig:Figure_2}
\end{figure}

As the domain of $f(x)$ is $[-4,4]$, the graph of $g(x)$ is subject to the same domain. This is shown by the red rectangle.

\subsection{Part (b)}
Let us define $g(x)$ in terms of $f(x)$:

\begin{center}
    Any point on the graph of $g(x)$ satisfies the following:
    \[
    (a,b) = (2x+1, 3f(x))
    \]
    Therefore, 
    \[
    g(2x+1) = 3f(x)
    \]
    Let $\phi = 2x+1$, so 
    \[
    g(\phi) = 3f(x)
    \]
    Now, express $x$ in terms of $\phi$:
    \[
    \phi = 2x+1
    \]
    \[
    \phi - 1 = 2x
    \]
    \[
    x = \frac{\phi - 1}{2}
    \]
    Substituting this back into the original equation:
    \[
    g(\phi) = 3f\left(\frac{\phi}{2} - \frac{1}{2}\right)
    \]
    Replacing $\phi$ with $x$, we get:
    \[
    g(x) = 3f\left(\frac{x}{2} - \frac{1}{2}\right)
    \]
    Thus, we have arrived at our solution.
\end{center}

\subsection{Part (c)}
Starting from the inside out, let's examine the effect of the transformation $ y = f\left(\frac{x}{2} - \frac{1}{2}\right)$.

\begin{figure}[H]
    \includegraphics[width = \linewidth]{Figure_3.png}
    \caption{Graph of $y = f\left(\frac{x}{2} - \frac{1}{2}\right)$ vs $y = f(x)$}
    \label{fig:Figure_3}
\end{figure}

It appears that the graph is horizontally compressed by a factor of 2, followed by a translation by the vector $(-\frac{1}{2}, 0)$. Now, if we plot only $y = f\left(\frac{x}{2}\right)$:

\begin{figure}[H]
    \includegraphics[width = \linewidth]{Figure_4.png}
    \caption{Graph of $y = f\left(\frac{x}{2}\right)$ vs $y = f(x)$}
    \label{fig:Figure_4}
\end{figure}

We see that the graph has only been compressed horizontally. Thus, scaling the input is inversely proportional to the horizontal stretch, and adding a constant translates the graph by a vector of $(c,0)$, where $c$ is the constant.

\vspace{3mm} Multiplying by three is fairly straghtforward. it stretches the graph vertically by a factor of 3. More precisely, it triples the function’s derivative.

\begin{center}
    Let $\phi = \frac{x}{2} - \frac{1}{2}$, and differentiate:
    \[
    \frac{d}{dx}\left[3f(\phi)\right] = 3 \frac{d}{dx}\left[f(\phi)\right] = 3 f'(\phi)
    \]
\end{center}

Therefore, the final transformation consists of a horizontal compression by a factor of 2, a leftward translation by $\frac{1}{2}$, and a vertical stretch by a factor of 3, leaving us with figure 5 below.

\begin{figure}[H]
    \includegraphics[width = \linewidth]{Figure_5.png}
    \caption{Graph of $y = f\left(\frac{x}{2}\right)$ vs $y = f(x)$}
    \label{fig:Figure_5}
\end{figure}

\end{document}
