\documentclass{article}
\usepackage{amsmath}
\usepackage{enumitem}

\title{Counting and Probability Week 6 Writing Problem}
\author{mxsail \\ Art of Problem Solving \\ Intermediate Counting and Probability}
\date{\today}

\begin{document}
\maketitle

\section*{Problem Statement}
\begin{enumerate}[label=(\alph*)]
    \item Prove that given any set of $5$ integers, there exist three of them whose sum is divisible by $3.$
    \item Prove that given any set of $17$ integers, there exist nine of them whose sum is divisible by $9.$
\end{enumerate}

\section*{Solution}

\subsection*{(a)}
Let the set of $5$ integers be 
\[
S = \{a_1, a_2, a_3, a_4, a_5\}.
\]
Consider their residues modulo $3$.  
Each integer has residue $0$, $1$, or $2$ (mod $3$).  

If any residue occurs at least $3$ times, then choosing those three integers gives a sum congruent to $0 \pmod{3}$, and we are done.

Otherwise, no residue appears more than twice. Since there are $5$ integers and $3$ residue classes, the only possible distribution of counts is $(2,2,1)$ in some order.  
In this case, all three residue classes are represented, so we can pick one integer of each residue. Their sum is
\[
0 + 1 + 2 \equiv 0 \pmod{3}.
\]
Hence, in every possible case, we can find three integers whose sum is divisible by $3$.

\[
\boxed{\text{Therefore, among any five integers, there exist three whose sum is divisible by } 3.}
\]

\subsection*{(b)}
This is a special case of a well-known result in additive combinatorics called the \textbf{Erd\H{o}s--Ginzburg--Ziv theorem}.

\medskip
\noindent\textbf{Theorem (Erd\H{o}s--Ginzburg--Ziv).}  
For any positive integer $n$, given any $2n - 1$ integers, there exists a subset of $n$ of them whose sum is divisible by $n$.

\medskip
Applying this theorem with $n = 9$, we find that among any $17$ integers, there exist $9$ whose sum is divisible by $9$.

\[
\boxed{\text{Therefore, among any 17 integers, there exist nine whose sum is divisible by } 9.}
\]

\end{document}
