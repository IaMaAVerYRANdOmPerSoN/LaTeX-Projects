\documentclass{article}
\usepackage{amsmath}
\usepackage{amssymb}

\title{Counting and Probablilty Week 4 Writing Problem}
\author{mxsail} % Non-capitialized name to align with mxsail's AoPS username
\date{\today}

\begin{document}
\maketitle

\section*{Problem Statement}
Consider a house with a finite number of rooms.
Prove that if every room has an even number of doors,
then the number of doors from the house to the outside world is also even.

\section*{Solution}
Let the house have \( n \) rooms, and let \( d_i \) be the number of doors in room \( i \) for \( i = 1, 2, \ldots, n \).
By the problem statement, we know that each \( d_i \) is even.
Becuase \( d_i \) is even, we know that \( \underset{j=1}{\overset{n}{\sum}} d_j \)
is also even, since the sum of even numbers is even.

We group the doors into two categories:
\begin{itemize}
    \item Doors that connect two rooms within the house.
    \item Doors that connect a room to the outside world.
\end{itemize}

By the problem statement, each door that connects two rooms is counted twice in the sum \( \underset{j=1}{\overset{n}{\sum}} d_j \),
once for each room it connects.
Let \( D \) be the number of doors that connect rooms to the outside world.
Then we can express the total sum of doors as:
\[\underset{j=1}{\overset{n}{\sum}} d_j = 2 \cdot (\text{number of internal doors}) + D\]
Because any number multiplied by 2 is even, \( 2 \cdot (\text{number of internal doors}) \) is even.
As the whole sum \( \underset{j=1}{\overset{n}{\sum}} d_j \) is even, it follows that \( D \) must also be even.
Thus, we have proven that if every room in the house has an even number of doors,
then the number of doors from the house to the outside world is also even.
\end{document}