\documentclass{article}
\usepackage{amssymb}
\usepackage{amsmath}

\title{Week 20 Writing Problem}
\author{mxsail, \\ Art Of Problem Solving,\\ Intermediate Algebra}
\date{\today}

\begin{document}

\maketitle

\section*{Problem Statement}
Let $x$ be a positive real number. Show that
\[
\frac{1}{x} \ge 3 - 2 \sqrt{x}.
\]
Describe when we have equality.

\section*{Solution}

We are asked to prove that for all $x > 0$,
\[
\frac{1}{x} \ge 3 - 2\sqrt{x}.
\]
We define the function
\[
f(x) = \frac{1}{x} + 2\sqrt{x} - 3,
\]
and we want to show that $f(x) \ge 0$ for all $x > 0$, and determine when equality occurs.

\subsection*{Step 1: Analyze the derivative}

First, compute the derivative of $f(x)$:
\[
f'(x) = \frac{d}{dx}\left(\frac{1}{x} + 2\sqrt{x} - 3\right) = -\frac{1}{x^2} + \frac{1}{\sqrt{x}}.
\]

Set the derivative equal to 0 to find critical points:
\[
-\frac{1}{x^2} + \frac{1}{\sqrt{x}} = 0
\Rightarrow \frac{1}{\sqrt{x}} = \frac{1}{x^2}
\Rightarrow x^2 = \sqrt{x}
\Rightarrow x^4 = x^{1/2}
\Rightarrow x^{7/2} = 1
\Rightarrow x = 1.
\]

So the only critical point in the domain $x > 0$ is at $x = 1$.

\subsection*{Step 2: Consider where the derivative is undefined}

The derivative $f'(x)$ is undefined at $x = 0$, but since $f(x)$ is only defined for $x > 0$, we are not concerned about $x = 0$ itself. However, we examine the behavior as $x \to 0^+$.

\subsection*{Step 3: Boundary behavior as $x \to 0^+$ and $x \to \infty$}

As $x \to 0^+$,
\[
\frac{1}{x} \to \infty, \quad \sqrt{x} \to 0, \quad \text{so } f(x) \to \infty.
\]

As $x \to \infty$,
\[
\frac{1}{x} \to 0, \quad \sqrt{x} \to \infty, \quad \text{so } f(x) \to \infty.
\]

\subsection*{Step 4: Evaluate the critical point}

We evaluate $f(1)$:
\[
f(1) = \frac{1}{1} + 2\sqrt{1} - 3 = 1 + 2 - 3 = 0.
\]

Since $f(x) \to \infty$ as $x \to 0^+$ and as $x \to \infty$, and the only critical point is a minimum at $x = 1$, we conclude that
\[
f(x) \ge 0 \quad \text{for all } x > 0,
\]
with equality if and only if $x = 1$.

\subsection*{Conclusion}

\[
\frac{1}{x} \ge 3 - 2\sqrt{x}
\]
for all $x > 0$, with equality if and only if $x = 1$.

\end{document}
