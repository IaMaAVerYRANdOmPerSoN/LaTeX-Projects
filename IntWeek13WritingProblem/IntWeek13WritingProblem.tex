\documentclass{article}
\usepackage{amsmath}
\usepackage{amssymb}
\usepackage{graphicx}
\usepackage{float}
\usepackage{hyperref}
\hypersetup{
    colorlinks=true,
    linkcolor=blue,
    filecolor=magenta,      
    urlcolor=cyan,
    pdftitle={Overleaf Example},
    pdfpagemode=FullScreen,
    }

\title{Trinomial roots and arithmetic progressions}
\author{Michael Xie, \\
        Art of Problem Solving, \\
        Intermediate Algebra}
\date{\today}

\begin{document}
\maketitle

\section{Problem}
\subsection{(a)}

Prove that if the roots of
\[x^3 + ax^2 + bx + c = 0\]
form an arithmetic sequence, then $2a^3 + 27c = 9ab.$

\subsection{(b)}
Prove that if $2a^3 + 27c = 9ab,$ then the roots of
\[x^3 + ax^2 + bx + c = 0\]
form an arithmetic sequence.

\section{Solution}
\subsection{(a)}
Let the roots of the polynomial be $r_1, r_2, r_3$. Since they are in arithmetic progression, we can express them as:
\[r_1 = r\]
\[r_2 = r + d\]
\[r_3 = r + 2d\]
where $r$ is the first term and $d$ is the common difference. By Vieta's formulas, we have:
\[
a = -(r_1 + r_2 + r_3) = -3r - 3d
\]
\[
b = r_1r_2 + r_2r_3 + r_3r_1 = 3r^2 + 6rd + 2d^2
\]
\[
c = -r_1r_2r_3 = -r(r + d)(r + 2d)
\]
Substituting these expressions into the equation $2a^3 + 27c = 9ab$, we get:
\begin{align} \label{eq:part_a}
    -2(3r + 3d)^3 - 27r(r + d)(r + 2d) &= -9(3r^2 + 6dr + 2d^2)(3r + 3d), \\
    -2 \cdot 27(r + d) ^3 - 27r(r + d)(r + 2d) &= -27(3r^2 + 6dr + 2d^2)(r + d), \\
    2(r + d)^3 + r(r + d)(r + 2d) &= (3r^2 + 6dr + 2d^2)(r + d), \\
    2(r + d)^2 + r(r + 2d) &= 3r^2 + 6dr + 2d^2, \\
    2(r^2 + 2dr + d^2) + r^2 + 2dr &= 3r^2 + 6dr + 2d^2, \\
    2r^2 + 4dr + 2d^2 + r^2 + 2dr &= 3r^2 + 6dr + 2d^2, \\
    3r^2 + 6dr + 2d^2 &= 3r^2 + 6dr + 2d^2.
\end{align}
This is an identity, which means that the equation holds for all values of $r$ and $d$. Thus, we have shown that if the roots of the polynomial are in arithmetic progression, then $2a^3 + 27c = 9ab$.

\subsection{(b)}
Prove that if $2a^3 + 27c = 9ab,$ then the roots of
\[x^3 + ax^2 + bx + c = 0\]
form an arithmetic sequence.
Assume that $2a^3 + 27c = 9ab$. We can express $c$ in terms of $a$ and $b$:
\[c = \frac{9ab - 2a^3}{27}\]
Using Vieta's formulas, we have:
\[
r_1 + r_2 + r_3 = -a
\]
\[
r_1r_2 + r_2r_3 + r_3r_1 = b
\]
\[
r_1r_2r_3 = -c
\]
Substituting the expression for $c$ into the third equation, we get:
\[
r_1r_2r_3 = -\frac{9ab - 2a^3}{27}
\]
Now, let us consider the general case for the roots $r_1, r_2, r_3$ of the cubic. By Vieta's formulas, we have:
\[
r_1 + r_2 + r_3 = -a, \quad r_1r_2 + r_2r_3 + r_3r_1 = b, \quad r_1r_2r_3 = -c.
\]
Given $2a^3 + 27c = 9ab$, substitute $c = \frac{9ab - 2a^3}{27}$ into the third Vieta equation:
\[
r_1r_2r_3 = -\frac{9ab - 2a^3}{27}.
\]
Now, consider the elementary symmetric sums $S_1 = r_1 + r_2 + r_3$, $S_2 = r_1r_2 + r_2r_3 + r_3r_1$, $S_3 = r_1r_2r_3$. The only way for the relation $2a^3 + 27c = 9ab$ to always hold is if the roots are in arithmetic progression (as shown in part \hyperref[eq:part_a]{(a)}). Therefore, the roots must be of the form $r, r+d, r+2d$ for some $r$ and $d$. Thus, the roots form an arithmetic sequence.

\section{Conclusion}
We have shown that if the roots of the polynomial $x^3 + ax^2 + bx + c = 0$ form an arithmetic sequence, then $2a^3 + 27c = 9ab$. Conversely, if $2a^3 + 27c = 9ab$, then the roots of the polynomial also form an arithmetic sequence. Thus, we have proven both parts of the problem.

\end{document}