\documentclass{article}

\usepackage{graphicx}
\usepackage{amssymb}
\usepackage{amsfonts}
\usepackage{amsmath}

\title{Solving Exponential Equations}
\author{Micheal Xie \\ Introduction to Algebra B, Art of Problem Solving}
\date{\today}

\begin{document}

\maketitle

\section{Introduction}

In this article, we aim to solve the exponential equation
\[
(3^x - 27)^3 + (27^x - 3)^3 = (3^x + 27^x - 30)^3.
\]
Our approach will use substitution and factoring.

\section{Solution}

\subsection{Substitution}

We begin by letting 
\[
t = 3^x.
\]
Since \(27^x = (3^3)^x = 3^{3x} = (3^x)^3 = t^3\), the equation becomes
\[
(t - 27)^3 + (t^3 - 3)^3 = (t + t^3 - 30)^3.
\]
Now, define
\[
a = t - 27 \quad \text{and} \quad b = t^3 - 3.
\]
Then our equation can be written as
\[
a^3 + b^3 = (a+b)^3.
\]
This key observation will allow us to factor the equation.

\subsection{Factoring}

Recall the expansion of a cube:
\[
(a+b)^3 = a^3 + b^3 + 3a^2b + 3ab^2.
\]
Substituting into our equation, we have
\[
a^3 + b^3 = a^3 + b^3 + 3a^2b + 3ab^2.
\]
Subtracting \(a^3+b^3\) from both sides yields
\[
3a^2b + 3ab^2 = 0.
\]
Factor out \(3ab\):
\[
3ab(a+b) = 0.
\]
Thus, at least one of the following must hold:
\[
a = 0,\quad b = 0, \quad \text{or} \quad a+b = 0.
\]

Rewriting these conditions in terms of \(t\):

\begin{enumerate}
    \item \(a = 0\): 
    \[
    t - 27 = 0 \implies t = 27.
    \]
    \item \(b = 0\): 
    \[
    t^3 - 3 = 0 \implies t^3 = 3 \implies t = \sqrt[3]{3}.
    \]
    \item \(a+b = 0\): 
    \[
    (t - 27) + (t^3 - 3) = 0 \implies t^3 + t - 30 = 0.
    \]
\end{enumerate}

We now focus on solving the third equation,
\[
t^3 + t - 30 = 0.
\]
Notice that \(t = 3\) is a solution since
\[
3^3 + 3 - 30 = 27 + 3 - 30 = 0.q
\]
To factor \(t^3 + t - 30\), perform polynomial division or factor by grouping. We can write:
\begin{align*}
t^3 + t - 30 &= t^3 - 9t + 10t - 30 \\
             &= t(t^2 - 9) + 10(t - 3) \\
             &= t(t-3)(t+3) + 10(t-3) \\
             &= (t-3)\bigl[t(t+3) + 10\bigr] \\
             &= (t-3)(t^2+3t+10).
\end{align*}
Since the quadratic \(t^2+3t+10\) has discriminant 
\[
\Delta = 3^2 - 4\cdot 1\cdot 10 = 9 - 40 = -31,
\]
its roots are non-real. Therefore, the only real solution from this factorization is 
\[
t - 3 = 0 \implies t = 3.
\]

\subsection{Back-Substitution}

Recall that \(t = 3^x\). The three real cases are:

\begin{enumerate}
    \item \(t = 27\):
    \[
    3^x = 27 \implies 3^x = 3^3 \implies x = 3.
    \]
    \item \(t = \sqrt[3]{3}\):
    \[
    3^x = \sqrt[3]{3} = 3^{1/3} \implies x = \frac{1}{3}.
    \]
    \item \(t = 3\):
    \[
    3^x = 3 \implies 3^x = 3^1 \implies x = 1.
    \]
\end{enumerate}

\section{Conclusion}

The real solutions to the exponential equation 
\[
(3^x - 27)^3 + (27^x - 3)^3 = (3^x + 27^x - 30)^3
\]
are 
\[
x = \frac{1}{3},\quad x = 1,\quad \text{and} \quad x = 3.
\]

\end{document}