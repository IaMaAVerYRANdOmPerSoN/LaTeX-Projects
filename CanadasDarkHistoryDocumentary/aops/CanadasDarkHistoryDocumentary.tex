% Preamble
\documentclass{article}
\usepackage{graphicx}

\title{Canada's Dark History: a Sumarised Testimony}
\author{Michael Xie, \\
Year 9, \\
CANADIAN HISTORY SINCE WORLD WAR I
}
\date{\today}
% Preamble ends

\begin{document}

\maketitle

\section{What type of atmosphere did Roberta describe the Mohawk Institute as having?}
Roberta expressed herself as a survivor of the Mohawk Institute, comparing the “school” to a military camp, stating it was run by fear. 

\section{How did Roberta’s experience at the Mohawk Institute affect her freedom as a child?}
She explained:
\begin{itemize}
    \item “I didn’t have any freedom as a child or a teenager.”
    \item "One of the nuns said to me: 'You will never leave here'”
\end{itemize}

\section{What realization did Roberta come to about her childhood experiences and the concept of residential schools?}
Roberta realized that her life was confined to the Mohawk Institute. She had zero understanding of the world outside of the confines of school, showing how residential schools methodically erased children’s connections. She was dumbfounded at the environment surrounding her after “graduating”, in essence a newborn baby, only at the age of 18 with a (almost) fully developed brain, unable to adapt to the new environment. She came to realize that she was in a limbo, dejected from both white and indigenous society, like a newborn baby whose parents died in tragedy, with no support to guide her into a normal life.

\section{Why [How] did the Canadian government justify the removal of Indigenous children from their homes?}
The government excused it as a “civilizing mission”, failing to observe indigenous culture, passing it off as a “savage” set of ideals, more of an obstacle obstructing the government’s objectives that something worth carefully considering .Officials claimed they were providing formal education and integrating Indigenous children into Euro-Canadian society, advertising how “effective” they were in the media, while in reality they were systematically erasing indigenous culture by isolating them from their roots and teaching them “Civility and discipline”, without legitimately introducing them to Euro-Canadian society and providing meaningful development pathways.

\section{What were some of the long-term impacts of residential schools on Indigenous communities?}
\begin{itemize}
    \item Loss of cultural values and lifestyle, including language and survival skills.
    \item Profound, Intergenerational trauma, often invisible, buried deep in time and survivor’s brains
    \item Mistrust of institutions, often too foundational to directly address.
    \item Loss of family and support structures. Many survivors and their descendants still do not know who their ancestors were, where they came from, and what they practised.
    \item Lasting effects on mental health, identity, and self-worth.
\end{itemize}

\section{How did the Canadian government’s policies towards Indigenous people reflect broader [European] colonial attitudes?}
These policies were founded in paradigm colonial belief of cultural superiority and “being god’s chosen one”. Indigenous peoples were treated as “savage” and in need of "civility". The government imposed elaborate and well-funded systems carefully designed to strip away any Indigenous identity and overshadow it with European values, enhancing control and domination rather than partnership or respect.

\section{What role does education play in preventing the erasure of Indigenous histories and cultures in Canada today?}
Education can:
\begin{itemize}
    \item Show the true history of residential schools and the 60’s scoop.
    \item Yield Indigenous voices, perspectives, and knowledge, gradually integrating applicable elements of their culture into modern society.
    \item Create room for empathy and understanding between damaged or destroyed communities and institutions.
    \item Support cultural revitalization and research on indigenous culture.
    \item Prevent denialism and ensure future generations remember and learn from the past.
\end{itemize}

\end{document}