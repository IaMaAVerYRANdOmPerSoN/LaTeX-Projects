\documentclass{article}
\usepackage{amsmath}
\usepackage{amssymb}

\title{Function invertability}
\author{Michael Xie, \\ AoPS Intermediate Algebra}        
\begin{document}

\maketitle

\section{Introduction}
In this article, we will discuss the invertibility of functions. A function $f : A \to B$ is said to have an inverse function if there exists a function $g : B \to A$ such that $f(g(x)) = x$ for all $x \in B$ and $g(f(x)) = x$ for all $x \in A$.

\section{Show that the Function $f(x) = x-\frac{1}{x}$ is not invertible}

Let $f : (-\infty,0) \cup (0,\infty) \to \mathbb{R}$ be defined by
\[ f(x) = x - \frac{1}{x}. \]
Show that $f$ has no inverse function.

\subsection*{Solution:}

To show that $f$ has no inverse function, we need to show that $f$ is not one-to-one. This means that there exist distinct $x_1$ and $x_2$ such that $f(x_1) = f(x_2)$.

Consider $f(x_1) = x_1 - \frac{1}{x_1}$ and $f(x_2) = x_2 - \frac{1}{x_2}$. Suppose $f(x_1) = f(x_2)$, then:
\[ x_1 - \frac{1}{x_1} = x_2 - \frac{1}{x_2}. \]

Rearranging terms, we get:
\[ x_1 - x_2 = \frac{1}{x_1} - \frac{1}{x_2}. \]

Multiplying both sides by $x_1 x_2$, we obtain:
\[ x_1 x_2 (x_1 - x_2) = x_2 - x_1. \]

This simplifies to:
\[ x_1 x_2 (x_1 - x_2) = -(x_1 - x_2). \]

If $x_1 \neq x_2$, we can divide both sides by $(x_1 - x_2)$:
\[ x_1 x_2 = -1. \]

Thus, for any $x_1 \in (-\infty,0) \cup (0,\infty)$, there exists $x_2 = -\frac{1}{x_1}$ such that $f(x_1) = f(x_2)$. Since $x_1 \neq x_2$, $f$ is not one-to-one and therefore does not have an inverse function.

\section{Show that the Function $g(x) = x-\frac{1}{x} : (0, \infty)$ is invertible}

Let $g : (0,\infty) \to \mathbb{R}$ be defined by
\[ g(x) = x - \frac{1}{x}. \]
Show that $g$ has an inverse function.

\subsubsection*{Solution:}

To show that $g$ has an inverse function, we need to show that $g$ is one-to-one and onto.

First, we show that $g$ is one-to-one. Suppose $g(x_1) = g(x_2)$ for $x_1, x_2 \in (0, \infty)$. Then:
\[ x_1 - \frac{1}{x_1} = x_2 - \frac{1}{x_2}. \]

Rearranging terms, we get:
\[ x_1 - x_2 = \frac{1}{x_1} - \frac{1}{x_2}. \]

Multiplying both sides by $x_1 x_2$, we obtain:
\[ x_1 x_2 (x_1 - x_2) = x_2 - x_1. \]

This simplifies to:
\[ x_1 x_2 (x_1 - x_2) = -(x_1 - x_2). \]

If $x_1 \neq x_2$, we can divide both sides by $(x_1 - x_2)$:
\[ x_1 x_2 = -1, \]

which is a contradiction since $x_1, x_2 > 0$. Therefore, $x_1 = x_2$, and $g$ is one-to-one.

Next, we show that $g$ is onto. For any $y \in \mathbb{R}$, we need to find $x \in (0, \infty)$ such that $g(x) = y$. Consider the equation:
\[ y = x - \frac{1}{x}. \]

Rearranging terms, we get:
\[ x^2 - yx - 1 = 0. \]

This is a quadratic equation in $x$, and its discriminant is:
\[ \Delta = y^2 + 4. \]

Since $\Delta > 0$ for all $y \in \mathbb{R}$, there are two real solutions for $x$:
\[ x = \frac{y \pm \sqrt{y^2 + 4}}{2}. \]

Since we are considering $x \in (0, \infty)$, we take the positive solution:
\[ x = \frac{y + \sqrt{y^2 + 4}}{2}. \]

Thus, for any $y \in \mathbb{R}$, there exists $x \in (0, \infty)$ such that $g(x) = y$. Therefore, $g$ is onto.

Since $g$ is both one-to-one and onto, it has an inverse function.

\end{document}