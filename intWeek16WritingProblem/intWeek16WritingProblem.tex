\documentclass{article}
\usepackage{amsmath}
\usepackage{amsfonts}
\usepackage{amssymb}

\title{Expression of logarithmic functions in terms of other logarithms}
\author{mxsail,\\ Art of Problem Solving,\\ Intermediate Algebra}
\date{\today}

\begin{document}

\maketitle

\section*{Introduction}
In this document, we will explore the expression of logarithmic functions in terms of other logarithms. We use an example to illustrate the process of rewriting logarithmic expressions.

\subsection*{Problem Statement}
Let $P = \log_8 3$ and $Q = \log_3 5$. Express $\log_{10} 5$ in terms of $P$ and $Q$. Your answer should no longer include any logarithms.

\section*{Solution 1}
To express $\log_{10} 5$ in terms of $P$ and $Q$, we start by using the change of base formula for logarithms:
\[
\log_{10} 5 = \frac{\log_3 5}{\log_3 10}
\]

Substituting $Q$ for $\log_3 5$, we have:
\[
\log_{10} 5 = \frac{Q}{\log_3 10}
\]

Now, we use the change of base formula again to express $\log_3 10$ in terms of $P$:
\[
\frac{Q}{\log_3 10} = \frac{Q}{\frac{\log_8 10}{\log_8 3}} = \frac{Q}{\frac{\log_8 10}{P}}
\]

This simplifies to:
\begin{align*}
    \log_{10} 5 &= \frac{QP}{\log_8 10}, \\
    &= \frac{QP}{\frac{1}{3} \log_2 10}, \\
    &= \frac{3QP}{\log_2 10}.
\end{align*}
Where $\log_2 10$ can be expressed as:
\begin{align*}
    \log_2 10 &= \log_2 (2 \cdot 5), \\
    &= \log_2 2 + \log_2 5, \\
    &= 1 + \log_2 5.
\end{align*}

Thus, we can express $\log_{10} 5$ as:
\[
\log_{10} 5 = \frac{3QP}{1 + \log_2 5}
\]

We can simplify further by expressing $\log_2 5$ in terms of $Q$ using the change of base formula:
\[
\log_2 5 = \frac{\log_3 5}{\log_3 2} = \frac{Q}{\log_3 2}
\]

And once again using the change of base formula:
\[
\log_3 2 = \frac{\log_8 2}{\log_8 3} = \frac{1/3}{P} = \frac{1}{3P}
\]

Substituting this into our expression for $\log_2 5$ gives:
\[
\log_2 5 = \frac{Q}{\frac{1}{3P}} = 3QP.
\]

Thus, we can express $\log_{10} 5$ as:
\[
\log_{10} 5 = \frac{3QP}{1 + 3QP}.
\]

Or, equivalently:
\[
\frac{3Q}{P + 3Q}
\]

\section*{Solution 2}
Another approach to express $\log_{10} 5$, $P$ and $Q$ in terms of the natural logarithm:
\[
\log_{10} 5 = \frac{\ln 5}{\ln 10}, \quad P = \frac{\ln 3}{\ln 8}, \quad Q = \frac{\ln 5}{\ln 3}.
\]
Using the product rule, we can express $\ln 10$ in terms of $P$ and $Q$:
\[
\ln 10 = \ln (2 \cdot 5) = \ln 2 + \ln 5.
\]
Substituting $\ln 5 = Q \ln 3$ and $\ln 2 = \frac{\ln 8}{3} = \frac{P \ln 3}{3}$, we have:
\[
\ln 10 = \frac{P \ln 3}{3} + Q \ln 3 = \left(\frac{P}{3} + Q\right) \ln 3.
\]
Thus, we can express $\log_{10} 5$ as:
\[
\log_{10} 5 = \frac{\ln{5}}{\left(\frac{P}{3} + Q\right)} = \frac{Q}{\left(\frac{P}{3} + Q\right)} = \frac{3Q}{P + 3Q}.
\]

\section*{Conclusion}
In this article, we have successfully expressed $\log_{10} 5$ in terms of $P$ and $Q$. We used the change of base formula and properties of logarithms to derive the final expression. The result is:
\[
\log_{10} 5 = \frac{3Q}{P + 3Q}.
\]

\end{document}
