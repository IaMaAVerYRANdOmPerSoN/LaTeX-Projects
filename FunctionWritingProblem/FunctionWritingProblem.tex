\documentclass{article}
\usepackage{amsmath}
\usepackage{amssymb}
\usepackage[a4paper,
            bindingoffset=0.2in,
            left=1in,
            right=1in,
            top=1in,
            bottom=1in,
            footskip=.25in]{geometry}

\title{Function Analysis}
\date{December 2024}
\author{Michael Xie \\ Art of Problem Solving} 

\begin{document}

\maketitle

\section*{Exploring the Differences Between Two Functions}

In this article, we explore the differences between the following two functions:
\[
f(x) = \sqrt{\frac{x+5}{x-7}} \quad \text{and} \quad h(x) = \frac{\sqrt{x+5}}{\sqrt{x-7}}.
\]

At first glance, these functions may seem very similar. However, there are key differences in their domains and ranges, which we will analyze in detail. 

\section*{Domain and Range Analysis}

\subsection*{Domain of $f(x)$}
The function $f(x) = \sqrt{\frac{x+5}{x-7}}$ is defined only when both of the following conditions hold:
\begin{itemize}
    \item The denominator $x - 7 \neq 0$ (to avoid division by zero).
    \item The expression $\frac{x+5}{x-7} \geq 0$ (to ensure the square root is defined and real).
\end{itemize}

\subsection*{Step 1: Avoiding Division by Zero}
The denominator $x - 7 \neq 0$ implies:
\[
x \neq 7.
\]

\subsection*{Step 2: Ensuring Non-Negativity}
To ensure $\frac{x+5}{x-7} \geq 0$, we analyze the signs of the numerator $(x+5)$ and denominator $(x-7)$ over different intervals of $x$:
\begin{itemize}
    \item When $x + 5 \geq 0$ and $x - 7 > 0$, the fraction is non-negative. This occurs when $x \geq 7$.
    \item When $x + 5 \leq 0$ and $x - 7 < 0$, the fraction is also non-negative. This occurs when $x \leq -5$.
\end{itemize}

Combining these conditions, the domain of $f(x)$ is:
\[
x \in (-\infty, -5] \cup [7, \infty).
\]

\subsection*{Step 3: Removing $x = 7$ (Division by Zero)}
Finally, removing $x = 7$ from the domain:
\[
\text{Domain of } f(x): \quad x \in (-\infty, -5] \cup (7, \infty) \subset \mathbb{R}.
\]

\subsection*{Domain of $h(x)$}
The function $h(x) = \frac{\sqrt{x+5}}{\sqrt{x-7}}$ imposes stricter constraints due to the square roots in both the numerator and denominator. It is defined only when:
\begin{itemize}
    \item The numerator $\sqrt{x+5}$ is defined, which requires $x + 5 \geq 0 \implies x \geq -5$.
    \item The denominator $\sqrt{x-7}$ is defined and nonzero, which requires $x - 7 > 0 \implies x > 7$.
\end{itemize}

Combining these conditions, the domain of $h(x)$ is:
\[
\text{Domain of } h(x): \quad x \in (7, \infty) \subset \mathbb{R}.
\]

\section*{Comparison of Domains}
The key difference between the domains of $f(x)$ and $h(x)$ lies in the inclusion of the interval $(-\infty, -5]$ for $f(x)$, which is not part of the domain of $h(x)$. This is because the two square roots in $h(x)$ impose stricter constraints, requiring both $x+5 \geq 0$ and $x-7 > 0$ simultaneously, compared to $f(x)$'s one, encompassing square root.

\end{document}
