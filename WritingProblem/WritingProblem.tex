\documentclass{article}
\usepackage{amsmath}
\usepackage{amssymb}
\usepackage[a4paper,
            bindingoffset=0.2in,
            left=1in,
            right=1in,
            top=1in,
            bottom=1in,
            footskip=.25in]{geometry}

\begin{document}

\section*{Solving the Inequality}
In this article, we aim to solve the inequality 
\[
\frac{x^2 + 2x + 5}{3x^2 - x - 4} \geq 0,
\]
using algebraic methods and casework analysis. By carefully handling the denominator and evaluating the behavior of the quadratic expressions, we will determine the solution set.

\subsection*{Step 1: Analyze the Inequality}
The original inequality is:
\[
\frac{x^2 + 2x + 5}{3x^2 - x - 4} \geq 0.
\]
To simplify, we multiply through by \( 3x^2 - x - 4 \). However, this requires casework because the inequality flips when \( 3x^2 - x - 4 < 0 \). Additionally, we must exclude points where the denominator is zero, as division by zero is undefined:
\[
3x^2 - x - 4 \neq 0.
\]
Thus, the inequality becomes:
\[
x^2 + 2x + 5 \geq 0, \quad \text{if } 3x^2 - x - 4 > 0,
\]
and:
\[
x^2 + 2x + 5 \leq 0, \quad \text{if } 3x^2 - x - 4 < 0.
\]

\subsection*{Step 2: Solve for \( 3x^2 - x - 4 = 0 \)}
We find the roots of \( 3x^2 - x - 4 \) by factoring:
\[
3x^2 - x - 4 = 0,
\]
\[
(3x - 4)(x + 1) = 0,
\]
\[
x = -1, \quad x = \frac{4}{3}.
\]
The quadratic \( 3x^2 - x - 4 \) changes sign at these points. We analyze the intervals:
\[
3x^2 - x - 4 =
\begin{cases}
< 0, & \text{if } x \in (-1, \frac{4}{3}), \\
> 0, & \text{if } x \notin [-1, \frac{4}{3}].
\end{cases}
\]

\subsection*{Step 3: Analyze \( x^2 + 2x + 5 \)}
We now examine the behavior of the numerator \( x^2 + 2x + 5 \):
\[
x^2 + 2x + 5 = 0.
\]
Using the quadratic formula:
\[
x = \frac{-2 \pm \sqrt{-16}}{2}.
\]
The discriminant \( \Delta = -16 \) is negative, so the quadratic has no real roots. Since the coefficient of \( x^2 \) is positive, the parabola opens upwards, meaning:
\[
x^2 + 2x + 5 > 0 \quad \text{for all } x \in \mathbb{R}.
\]

\subsection*{Step 4: Case Analysis}
\subsubsection*{Case 1: \( 3x^2 - x - 4 > 0 \)}
When \( 3x^2 - x - 4 > 0 \), the inequality reduces to:
\[
x^2 + 2x + 5 \geq 0.
\]
Since \( x^2 + 2x + 5 > 0 \) for all \( x \), this case holds true for all \( x \notin [-1, \frac{4}{3}] \).

\subsubsection*{Case 2: \( 3x^2 - x - 4 < 0 \)}
When \( 3x^2 - x - 4 < 0 \), the inequality becomes:
\[
x^2 + 2x + 5 \leq 0.
\]
However, as previously established, \( x^2 + 2x + 5 > 0 \) for all \( x \). Therefore, this case does not contribute any solutions.

\subsection*{Step 5: Exclude Undefined Points}
We must exclude points where \( 3x^2 - x - 4 = 0 \), i.e., \( x = -1 \) and \( x = \frac{4}{3} \), as these make the denominator undefined.

\subsection*{Solution}
Combining the results from both cases, the solution to the inequality is:
\[
x \in (-\infty, -1) \cup \left(\frac{4}{3}, \infty\right).
\]

\subsection*{Conclusion}
By analyzing the numerator and denominator of the given inequality, we determined that \( x^2 + 2x + 5 > 0 \) for all \( x \), while the denominator \( 3x^2 - x - 4 \) changes sign at \( x = -1 \) and \( x = \frac{4}{3} \). Using casework, we excluded intervals where the denominator is negative or undefined. Thus, the final solution is:
\[
x \in (-\infty, -1) \cup \left(\frac{4}{3}, \infty\right).
\]

\end{document}